\documentclass[12pt]{article}

\usepackage{bbding}
\usepackage{amsthm}
\usepackage{amssymb}
\usepackage{amsmath}
\usepackage{amsfonts}

\author{Thomas O'Shaughnessy}
\title{Newton's Method}

\begin{document}
\maketitle
\section{Definitions}

\textbf{Newton's Method} is a root finding algorithm that converges for functions 
that satisfy certain
conditions. Beginning with a guess for a root (the closer the better), our implementation
of the algorithm will apply an iterative process to find an approximation of this root
to an arbitrary precision. The purpose of this paper is analyze how the algorithm behaves
for different functions and starting points (guesses).

Suppose we have an iterative process, which we will call a \textbf{Newton Transform}, defined as
\begin{gather*}
				x \mapsto x - \frac{f(x)}{f'(x)}.
\end{gather*}
Our first condition should thus be that $f'(x_i) \neq 0$, for any iteration $x_i$.
Our Newton Transform can be intuitively derived from the familiar slope definition
\begin{gather*}
				m=\frac{y-y_0}{x-x_0}.
\end{gather*}
After doing the appropriate substitutions for our linear approximation we have
\begin{gather*}
				f'(x_0)\approx\frac{f(x)-f(x_0)}{x-x_0},
\end{gather*}
where $x_0$ may be our initial guess. Notice that this expression can be rearranged
to represent the \textbf{1st Taylor polynomial}
\begin{gather*}
				f(x)\approx f(x_0)+f'(x_0)(x-x_0).
\end{gather*}
For now, however, we are only interested in the value of $x$ when our
linear approximation is $0$. Solving for $x$ with $f(x) = 0$, we have our Newton Transform.

\end{document}
