\documentclass[12pt]{article}

\usepackage{bbding}
\usepackage{amsthm}
\usepackage{amssymb}
\usepackage{amsmath}
\usepackage{amsfonts}

\author{Thomas O'Shaughnessy}
\title{Meeting 2}

\begin{document}
\maketitle

Consider some iterated function $f(x)$ such that some $x$ is a periodic point. If $f^{n+1}(x)=f(x)$,
we will say that $\mathcal{O}$ of $x$ has a periodicity of $n$. We will refer to the orbit of $x$,
the sequence of values of $f^m(x)$, as $\mathcal{O}_f$ and the periodicity of $\mathcal{O}_f$ as $n_f$.\\\\
Let $f(x) = \frac{1}{x}$. Notice that $n_f = 2$. Now let $g(x) = x - \frac{f(x)}{f'(x)}$. Thus, $g$ is
our Newton Transform. I will now show that $n_g = n_f$, for all $x\in\textup{Dom}(f)$. Consider the
following differential equation.
\begin{gather*}
				\frac{1}{x} = x - \frac{dx}{dy}y
\end{gather*}
This can be interpreted as a Newton Transform that has the same effect as taking the reciprocal
of $x$. It should seem reasonable that the periodicity of $f(x)$, on the left,
determines the periodicity o

\end{document}
